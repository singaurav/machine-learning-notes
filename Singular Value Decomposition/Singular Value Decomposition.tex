\documentclass[11pt, a4paper]{article}

\usepackage{tikz}
\usetikzlibrary{shapes,arrows}
\usepackage{amsmath}
\usepackage{placeins}
\usepackage{amssymb}

\begin{document}

\title{SINGULAR VALUE DECOMPOSITION}
\date{}
\maketitle

Singular Value Decomposition (SVD) is the factorization of any $m\times n$ matrix $A$ as $A = U \Sigma V^T$ where $U$ is a $m \times m$ matrix made up of columns of mutually perpendicular unit vectors, $\Sigma$ is a diagonal matrix of the same shape as $A$ and $V^T$ is a $n \times n$ matrix made up of rows of mutually perpendicular unit vectors. 

\section{Example}

Let $A$ be a $2 \times 3$ matrix as follows,

\begin{align*}
	A = \begin{pmatrix}
	3  & 0 & 0 \\
	-8 & 0 & 3 
	\end{pmatrix}
\end{align*}

To calculate unit vectors of $U$, eigenvalues and eigenvectors of $AA^T$ are calculated,

\begin{align*}
	AA^T &= \begin{pmatrix}
	3  & 0 & 0 \\
	-8 & 0 & 3 
	\end{pmatrix} \begin{pmatrix}
	3 & -8 \\
	0 & 0 \\
	0 & 3
	\end{pmatrix} \\
	&= \begin{pmatrix}
	9 & -24 \\
	-24 & 73
	\end{pmatrix}
\end{align*}

$AA^T - \lambda I$ must be singular if eigenvectors are non-zero. Hence,

\begin{align*}
	\begin{vmatrix}
	9 - \lambda             & -24          \\
	-24                     & 73 - \lambda 
	\end{vmatrix}           & = 0          \\
	(9-\lambda)(73-\lambda) & = 576        
\end{align*}

This gives $\lambda = 81, 1$ and $\hat{u} = \begin{pmatrix}
\frac{1}{\sqrt{10}} \\
\frac{-3}{\sqrt{10}}
\end{pmatrix},\begin{pmatrix}
\frac{3}{\sqrt{10}} \\
\frac{1}{\sqrt{10}}
\end{pmatrix}$ respectively. 

At this step, decomposition is as follows:

\begin{align*}
	A = (\hat{u_1}, \hat{u_2}) \begin{pmatrix}
	\sqrt{\lambda_1} & 0                & 0 \\
	0                & \sqrt{\lambda_2} & 0 
	\end{pmatrix} V^T \\
	A = \begin{pmatrix}
	\frac{1}{\sqrt{10}} & \frac{3}{\sqrt{10}} \\
	\frac{-3}{\sqrt{10}} & \frac{1}{\sqrt{10}}
	\end{pmatrix}      \begin{pmatrix}
	9                & 0                & 0 \\
	0                & 1                & 0 
	\end{pmatrix} V^T 
\end{align*}

To continue and calculate $V$, the relation $A\hat{v_i} = \sqrt{\lambda_i}\hat{u_i}$ is used. It is clear that $V$ is a $3 \times 3$ matrix.

\begin{align*}
	A\hat{v_1} &= \sqrt{\lambda_1}\hat{u_1} \\
	\begin{pmatrix}
	3  & 0 & 0 \\
	-8 & 0 & 3 
	\end{pmatrix}\hat{v_1} &= \sqrt{81}\begin{pmatrix}
	\frac{1}{\sqrt{10}} \\
	\frac{-3}{\sqrt{10}}
	\end{pmatrix} \\
	\hat{v_1} &= \begin{pmatrix}
	\frac{3}{\sqrt{10}} \\
	0 \\
	\frac{-1}{\sqrt{10}} \\
	\end{pmatrix}
\end{align*}

Similarly,

\begin{align*}
	A\hat{v_2} &= \sqrt{\lambda_2}\hat{u_2} \\
	\begin{pmatrix}
	3  & 0 & 0 \\
	-8 & 0 & 3 
	\end{pmatrix}\hat{v_2} &= \sqrt{1}\begin{pmatrix}
	\frac{3}{\sqrt{10}} \\
	\frac{1}{\sqrt{10}}
	\end{pmatrix} \\
	\hat{v_2} &= \begin{pmatrix}
	\frac{1}{\sqrt{10}} \\
	0 \\
	\frac{3}{\sqrt{10}} \\
	\end{pmatrix}
\end{align*}

Notice that $\hat{v_1}.\hat{v_2}$ is zero. The choices for $\hat{v_3}$ which must be perpendicular to both $\hat{v_1}$ and $\hat{v_2}$ are $(0, \pm 1, 0)^T$ and both are admissible in the decomposition as follows,

\begin{align*}    
	A &= \begin{pmatrix}
	\frac{1}{\sqrt{10}} & \frac{3}{\sqrt{10}} \\
	\frac{-3}{\sqrt{10}} & \frac{1}{\sqrt{10}}
	\end{pmatrix}      \begin{pmatrix}
	9                    & 0                   & 0                    \\
	0                    & 1                   & 0                    
	\end{pmatrix} \begin{pmatrix}
	\hat{v_1}            & \hat{v_2}           & \hat{v_3}            
	\end{pmatrix}^T \\
	A &= \begin{pmatrix}
	\frac{1}{\sqrt{10}} & \frac{3}{\sqrt{10}} \\
	\frac{-3}{\sqrt{10}} & \frac{1}{\sqrt{10}}
	\end{pmatrix}      \begin{pmatrix}
	9                    & 0                   & 0                    \\
	0                    & 1                   & 0                    
	\end{pmatrix} \begin{pmatrix}
	\frac{3}{\sqrt{10}}  & \frac{1}{\sqrt{10}} & 0                    \\
	0                    & 0                   & \pm 1                \\
	\frac{-1}{\sqrt{10}} & \frac{3}{\sqrt{10}} & 0                    \\              
	\end{pmatrix}^T \\
	\begin{pmatrix}
	3                    & 0                   & 0                    \\
	-8                   & 0                   & 3                    
	\end{pmatrix} &= \begin{pmatrix}
	\frac{1}{\sqrt{10}} & \frac{3}{\sqrt{10}} \\
	\frac{-3}{\sqrt{10}} & \frac{1}{\sqrt{10}}
	\end{pmatrix}      \begin{pmatrix}
	9                    & 0                   & 0                    \\
	0                    & 1                   & 0                    
	\end{pmatrix} \begin{pmatrix}
	\frac{3}{\sqrt{10}}  & 0                   & \frac{-1}{\sqrt{10}} \\
	\frac{1}{\sqrt{10}}  & 0                   & \frac{3}{\sqrt{10}}  \\              
	0                    & \pm 1               & 0                    \\    
	\end{pmatrix}
\end{align*}

\section{Interpretation}

\end{document}